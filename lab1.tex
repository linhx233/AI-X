\documentclass{article}
\usepackage{amsmath,amssymb,amsthm,graphicx,setspace,fullpage,verbatim,listings,xcolor}

\onehalfspace

\title{\bf\huge AI+X: Report 1}
\author{Hanxi Lin}

\lstset{
	backgroundcolor=\color{gray!10},   % 背景色
	basicstyle=\ttfamily\footnotesize, % 基本字体样式
	breaklines=true,                  % 自动换行
	frame=noframe,                     % 边框样式
	numbers=left,                     % 行号位置
	numberstyle=\tiny\color{gray},    % 行号样式
	keywordstyle=\color{blue},        % 关键字样式
	commentstyle=\color{green!50!black}, % 注释样式
	stringstyle=\color{orange},       % 字符串样式
	showstringspaces=false            % 不显示字符串中的空格
}

\begin{document}
\maketitle

\section*{Task 1}
Run the test command, we have the following output in \verb`network_stats.txt`:
\begin{lstlisting}[language={}]  
packets_injected = 5                       (Unspecified)
packets_received = 2                       (Unspecified)
average_packet_queueing_latency = 1000                       (Unspecified)
average_packet_network_latency = 3000                       (Unspecified)
average_packet_latency = 4000                       (Unspecified)
average_hops = 1.500000                       (Unspecified)
\end{lstlisting}
After changing the global frequency to \verb|2GHz|, , we have the following output in \verb`network_stats.txt`:
\begin{lstlisting}[language={}]
packets_injected = 3165                       (Unspecified)
packets_received = 3162                       (Unspecified)
average_packet_queueing_latency = 2                       (Unspecified)
average_packet_network_latency = 13.557559                       (Unspecified)
average_packet_latency = 15.557559                       (Unspecified)
average_hops = 5.269450                       (Unspecified)
\end{lstlisting}

\section*{Task 2}
By modifying the bash command to
\begin{lstlisting}[language=bash]
#! /bin/bash

./build/NULL/gem5.opt \
configs/example/garnet_synth_traffic.py \
--network=garnet --num-cpus=64 --num-dirs=64 \
--topology=Mesh_XY --mesh-rows=8 \
--inj-vnet=0 --synthetic=uniform_random \
--sim-cycles=10000 --injectionrate=0.01 

echo > network_stats.txt
grep "packets_injected::total" m5out/stats.txt | sed 's/system.ruby.network.packets_injected::total\s*/packets_injected = /' >> network_stats.txt
grep "packets_received::total" m5out/stats.txt | sed 's/system.ruby.network.packets_received::total\s*/packets_received = /' >> network_stats.txt
grep "average_packet_queueing_latency" m5out/stats.txt | sed 's/system.ruby.network.average_packet_queueing_latency\s*/average_packet_queueing_latency = /' >> network_stats.txt
grep "average_packet_network_latency" m5out/stats.txt | sed 's/system.ruby.network.average_packet_network_latency\s*/average_packet_network_latency = /' >> network_stats.txt
grep "average_packet_latency" m5out/stats.txt | sed 's/system.ruby.network.average_packet_latency\s*/average_packet_latency = /' >> network_stats.txt
grep "average_hops" m5out/stats.txt | sed 's/system.ruby.network.average_hops\s*/average_hops = /' >> network_stats.txt
\end{lstlisting}
We added the \verb|reception_rate| metric to \verb|network_stats.txt|. And by adding \verb|.unit()| in \verb|GarnetNetwork::regStats()|, we completed the units of statistics. In addition, there is an error in \verb|GarnetSyntheticTraffic.cc|: the phrase \verb|curTick() >= simCycles| should be \verb|curCycle() >= simCycles| The output in \verb|network_stats.txt| is as follows:
\begin{lstlisting}[language={}]
packets_injected = 6296                       (Count)
packets_received = 6290                       (Count)
reception_rate = .009828                       (Packet/(Node*Cycle))
average_packet_queueing_latency = 2                       (Tick)
average_packet_network_latency = 13.558188                       (Tick)
average_packet_latency = 15.558188                       (Tick)
average_hops = 5.269952                       (Count)
\end{lstlisting}

\section*{Task 3}
1.\begin{itemize}
	\item \verb|-h, --help|: show help message and exit.
	\item \verb|-n NUM_CPUS, --num-cpus NUM_CPUS|: Number of CPUs.
	\item \verb|--sys-voltage SYS_VOLTAGE|: Top-level voltage for blocks running at system power supply
	\item \verb|--sys-clock SYS_CLOCK|: Top-level clock for blocks running at system speed.
	\item \verb|--list-mem-types|: List available memory types.
	\item \begin{verbatim}--mem-type {CfiMemory,DDR3_1600_8x8,DDR3_2133_8x8,DDR4_2400_16x4,DDR4_2400_4x16,DDR4_2400
	_8x8,DDR5_4400_4x8,DDR5_6400_4x8,DDR5_8400_4x8,DRAMInterface,GDDR5_4000_2x32,HBM_1000_4H
	_1x128,HBM_1000_4H_1x64,HBM_2000_4H_1x64,HMC_2500_1x32,LPDDR2_S4_1066_1x32,LPDDR3_1600_1
	x32,LPDDR5_5500_1x16_8B_BL32,LPDDR5_5500_1x16_BG_BL16,LPDDR5_5500_1x16_BG_BL32,LPDDR5_64
	00_1x16_8B_BL32,LPDDR5_6400_1x16_BG_BL16,LPDDR5_6400_1x16_BG_BL32,NVMInterface,NVM_2400_
	1x64,QoSMemSinkInterface,SimpleMemory,WideIO_200_1x128}\end{verbatim}
	type of memory to use
	\item \verb|--mem-channels MEM_CHANNELS|: number of memory channels
	\item \verb|--mem-ranks MEM_RANKS|: number of memory ranks per channel
	\item \verb|--mem-size MEM_SIZE|: Specify the physical memory size (single memory)
	\item \verb|--enable-dram-powerdown|: Enable low-power states in DRAMInterface
	\item \verb|--mem-channels-intlv MEM_CHANNELS_INTLV|: Memory channels interleave
	\item \verb|--memchecker|
	\item \verb|--external-memory-system EXTERNAL_MEMORY_SYSTEM|: use external ports of this \verb|port_type| for caches
	
	\item \verb|--tlm-memory TLM_MEMORY|: use external port for SystemC TLM cosimulation
	\item \verb|--caches|
	\item \verb|--l2cache|
	\item \verb|--num-dirs NUM_DIRS|
	\item \verb|--num-l2caches NUM_L2CACHES|
	\item \verb|--num-l3caches NUM_L3CACHES|
	\item \verb|--l1d_size L1D_SIZE|
	\item \verb|--l1i_size L1I_SIZE|
	\item \verb|--l2_size L2_SIZE|
	\item \verb|--l3_size L3_SIZE|
	\item \verb|--l1d_assoc L1D_ASSOC|
	\item \verb|--l1i_assoc L1I_ASSOC|
	\item \verb|--l2_assoc L2_ASSOC|
	\item \verb|--l3_assoc L3_ASSOC|
	\item \verb|--cacheline_size CACHELINE_SIZE|
	\item \verb|--ruby|
	\item \verb|-m TICKS, --abs-max-tick TICKS|: Run to absolute simulated tick specified including ticks from a restored checkpoint
	\item \verb|--rel-max-tick TICKS|: Simulate for specified number of ticks relative to the simulation start tick (e.g. if restoring a checkpoint)
	\item \verb|--maxtime MAXTIME|: Run to the specified absolute simulated time in seconds
	\item \verb|-P PARAM, --param PARAM|: Set a \verb|SimObject parameter| relative to the root node. An extended Python multi range slicing syntax can be used for arrays. 
	\item \begin{verbatim}
	--synthetic {uniform_random,tornado,bit_complement,bit_reverse,bit_rotation,neighbor,
	shuffle,transpose}\end{verbatim}
	\item \verb|-i I, --injectionrate I|: Injection rate in packets per cycle per node. Takes decimal value between 0 to 1 (eg. 0.225). Number of digits after 0 depends upon \verb|--precision|.
	\item \verb|--precision PRECISION|: Number of digits of precision after decimal point for injection rate
	\item \verb|--sim-cycles SIM_CYCLES|: Number of simulation cycles
	\item \verb|--num-packets-max NUM_PACKETS_MAX|: Stop injecting after --num-packets-max. Set to -1 to disable.
	\item \verb|--single-sender-id SINGLE_SENDER_ID|: Only inject from this sender. Set to -1 to disable.
	\item \verb|--single-dest-id SINGLE_DEST_ID|: Only send to this destination. Set to -1 to disable.
	\item \verb|--inj-vnet {-1,0,1,2}|: Only inject in this vnet (0, 1 or 2). 0 and 1 are 1-flit, 2 is 5-flit. Set to -1 to inject randomly in all vnets.
	\item \verb|--ruby-clock RUBY_CLOCK|: Clock for blocks running at Ruby system's speed
	\item \verb|--access-backing-store|: Should ruby maintain a second copy of memory
	\item \verb|--ports PORTS|: used of transitions per cycle which is a proxy for the number of ports.
	\item \verb|--numa-high-bit NUMA_HIGH_BIT|: high order address bit to use for numa mapping. 0 = highest bit, not specified = lowest bit
	\item \verb|--interleaving-bits INTERLEAVING_BITS|: number of bits to specify interleaving in directory, memory controllers and caches. 0 = not specified
	\item \verb|--xor-low-bit XOR_LOW_BIT|: hashing bit for channel selection. See MemConfig for explanation of the default parameter. If set to 0, \verb|xor_high_bit| is also set to 0.

	\item \verb|--recycle-latency RECYCLE_LATENCY|: Recycle latency for ruby controller input buffers

	\item \verb|--topology TOPOLOGY|: check configs/topologies for complete set
	\item \verb|--mesh-rows MESH_ROWS|: the number of rows in the mesh topology
	\item \verb|--network {simple,garnet}|: 'simple'|'garnet' (garnet2.0 will be deprecated.)
	\item \verb|--router-latency ROUTER_LATENCY|: number of pipeline stages in the garnet router. Has to be >= 1. Can be over-ridden on a per router basis in the topology file.
	\item \verb|--link-latency LINK_LATENCY|: latency of each link the simple/garnet networks. Has to be >= 1. Can be over-ridden on a per link basis in the topology file.
	\item \verb|--link-width-bits LINK_WIDTH_BITS|: width in bits for all links inside garnet.
	\item \verb|--vcs-per-vnet VCS_PER_VNET|: number of virtual channels per virtual network inside garnet network.
	\item \verb|--routing-algorithm ROUTING_ALGORITHM|: routing algorithm in network. 0: weight-based table 1: XY (for Mesh. see garnet/RoutingUnit.cc) 2: Custom (see garnet/RoutingUnit.cc)
	\item \verb|--network-fault-model|: enable network fault model: see \verb|src/mem/ruby/network/fault_model/|
	\item \verb|--garnet-deadlock-threshold GARNET_DEADLOCK_THRESHOLD|: network-level deadlock threshold.
	\item \verb|--simple-physical-channels|: SimpleNetwork links uses a separate physical channel for each virtual network
\end{itemize}
The default values are defined in \verb|configs/common/Options.py| and \verb|configs/example/garnet_synth_traffic.py|.

2. The unit of \verb|sim-cycles| is cycles and the unit of \verb|router_latency,link_latency| is ticks. Cycles and ticks are related by the clock period. By default, 1 clock cycle is 1ns, and 1 tick is 1ps. The \verb|setGlobalFrequency()| function sets the tick frequency.

3. Packets per node per cycle. 

4. \verb|GarnetNetworkInterface| is defined in \verb|src/mem/ruby/network/garnet/GarnetNetworkInterface.hh| and implemented in \verb|src/mem/ruby/network/garnet/GarnetNetworkInterface.cc|. \verb|Router| is defined in \verb|src/mem/ruby/network/garnet/Router.hh| and implemented in \verb|src/mem/ruby/network/garnet/Router.cc|.

5. The packets are generated and injected into the network in \verb|GarnetSyntheticTraffic|, buffered in \verb|VirtualChannel| and \verb|GarnetNetworkInterface|. In \verb|SwitchAllocator| and \verb|CrossbarSwitch|, the program determines if packets can be sent downstream.
\end{document}